\section{Distributed Systems} %TODO
\label{sec:distributed-systems}

\subsection{Challenges in Distributed Systems}


A Distributed System does inherently not have a centralized, reliable source of truth, which creates a set of challenges that researchers have long tried to mitigate. One well-known example is the CAP theorem: \footnote{Related to this is also the PACELC theorem which differentiates between a partitioned and non-partitioned system} It states that a distributed system can not be consistent, available and partition-tolerant simultaneously. 
Issues of this type become more pronounced in a fog environment. There, computers are not just communicating at large scale -- as in a datacenter -- but additionally face increasingly unreliable infrastructure such as network or power outages.

% Rephrase this better -> maybe this also belongs in my own part?
% For the aforementioned established deployments this means, that assumptions on resources can be made and one of two cases is expected: 
% Either enough resources are available for a request to be executed instantly or requests are queued. From an architecture perspective this is reasonable since there is only one system layer capable of request execution, and we expect it to provide sufficient resources to handle the load.
% Contrarily, for FaaS in a fog environment, a request is forwarded instead of queued, since there are multiple serving layers, and it can not be assumed that a particular one is able to handle all load. 

