\documentclass[conference,10pt,letterpaper]{IEEEtran}

\usepackage{cite}
\usepackage{amsmath,amssymb,amsfonts}
\usepackage{algorithmic}
\usepackage{graphicx}
\usepackage{textcomp}
\usepackage{xcolor}
\usepackage{balance}
\usepackage[all]{nowidow}

\usepackage[hidelinks]{hyperref}
\usepackage[capitalise]{cleveref}

\begin{document}

\title{AuctionWhisk: Seeing Through the Fog}

\author{\IEEEauthorblockN{Example Student}
    \IEEEauthorblockA{\textit{Technische Universit\"at Berlin}\\
        example.student@campus.tu-berlin.de}
}

\maketitle

% Discuss the main advantages and disadvantages of AuctionWhisk. When is AuctionWhisk particularly effective? 
% Are there scenarios, in which AuctionWhisk is impossible to use?

\begin{abstract}
% In practice Function-as-a-Service (FaaS) deployments are often optimized to run in large datacenters, but as researchers and industries attempt to reduce latency further, the attention has shifted towards FogAuctionWhisk presents a approach to adapting the FaaS paradigm to the heterogeneous infrastructure of fog-computing by allowing developers to individually define the importance of application performance by bidding on resources. This not only allows for more efficient use of computation and improved overall system performance but also for increased earning of the platform provider.

Why does everything break
\end{abstract}

\section{Introduction}
\label{sec:introduction}

Function-as-a-service (FaaS), often also referred to as serverless computing, is a relatively new paradigm, where code is only run on demand in isolated environments. 
This makes it possible to bill users in a fine-grained manner, depending on metrics like number of executions or execution time. \cite{10.1145/3352460.3358296}

In practice FaaS deployments are often optimized to either run in large datacenters of big public cloud providers \cite{288770}, \cite{9302817} (from here on out also referred to as Cloud) or in designated edge-distributed networks\footnote{For example, Cloudflare's Worker platform is deployed on their globally distributed CDN}. 
However, as researchers and industries attempt to improve performance and reduce latency further, the attention has also shifted towards alternatives. 

One of those potential alternatives is Fog Computing, which we will here define as a fusion of Cloud and Edge Computing where computation resources are deployed on the network paths in between as to split up load over a more diverse infrastructure. 
That is, each layer between the Cloud and the Edge offers a unique trade-off between resource availability and latency. \cite{9326369}

Efficiently leveraging the distributed and heterogeneous resources of the aforementioned infrastructure, creates a new set of issues in addition to the preexisting challenges in distributed systems: 
For instance, determining appropriate scheduling algorithms or optimal allocation of available resources. 
AuctionWhisk proposed a new way of solving those issues by using an auction inspired approach. 
Auctions allow for stateless function scheduling without making assumptions on a centralized source of truth by creating a self-contained environment based on developer bids for a function. 
As in a real auction, the platform can determine the functions that should be scheduled -- respective winners of the auction -- based on those bids.


In this essay, we make the following contributions:
\begin{enumerate}
    \item We discuss the advantages and disadvantages of using auctions to determine function placement and execution.
    \item We evaluate applications that could potentially benefit or suffer from this approach.
\end{enumerate}



\section{Background and Related Work}
\label{sec:background}

Although, there has been much research on FaaS and the paradigm has arrived in mainstream technology, with many providers creating their own offering. Be it running in large datacenters (e.g. AWS Lambda, Google Cloud Run Functions, Microsoft Azure Functions) \cite{8241104} or distributed at the edge (Cloudflare Workers, Fastly Compute) \cite{10643918}. At the same time, there are only few commercial offerings for FaaS on fog infrastructure \cite{8814084} and research on the topic is also more scarce.

Deployments in large datacenters and at the edge are architecturally considerably different and offer their respective set of challenges. Nevertheless, one can make assumptions about resource availability in the corresponding environments. To illustrate, a deployment in the Cloud is expected to have virtually infinite, mostly reliable resources \cite{288770} but a deployment at the edge is expected to have very little, often unreliable resources \cite{9657141}. However, fog computing resources are often heterogeneous, with servers closer to the Edge typically possessing lower computational capacity. Consequently, we cannot make the same assumptions about resources here, as we could above. \cite{10.1145/2677046.2677052}

Furthermore, FaaS is traditionally run at a single layer and although requests may be routed through multiple load balancers, proxies, etc., they are only processed at one level of the call graph. %needs cite
Fog computing, on the other hand, commonly uses resources as proxies and servers \footnote{Server is a broad term, but here it intends to describe a computing instance that serves requests} simultaneously. 
This is often implemented by checking resource availability at request arrival and forwarding it if resources are insufficient for execution. \cite{https://doi.org/10.1002/spe.3058}

In this article, we propose two contributions:
\begin{enumerate}
    \item This is my first great idea
    \item This is my second great idea
\end{enumerate}



% Rephrase this better -> maybe this also belongs in my own part?
% For the aforementioned established deployments this means, that assumptions on resources can be made and one of two cases is expected: 
% Either enough resources are available for a request to be executed instantly or requests are queued. From an architecture perspective this is reasonable since there is only one system layer capable of request execution, and we expect it to provide sufficient resources to handle the load.
% Contrarily, for FaaS in a fog environment, a request is forwarded instead of queued, since there are multiple serving layers, and it can not be assumed that a particular one is able to handle all load. 


\section{Distributed Systems} %TODO
\label{sec:distributed-systems}

\subsection{Challenges in Distributed Systems}


A Distributed System does inherently not have a centralized, reliable source of truth, which creates a set of challenges that researchers have long tried to mitigate. One well-known example is the CAP theorem: \footnote{Related to this is also the PACELC theorem which differentiates between a partitioned and non-partitioned system} It states that a distributed system can not be consistent, available and partition-tolerant simultaneously. 
Issues of this type become more pronounced in a fog environment. There, computers are not just communicating at large scale -- as in a datacenter -- but additionally face increasingly unreliable infrastructure such as network or power outages.

% Rephrase this better -> maybe this also belongs in my own part?
% For the aforementioned established deployments this means, that assumptions on resources can be made and one of two cases is expected: 
% Either enough resources are available for a request to be executed instantly or requests are queued. From an architecture perspective this is reasonable since there is only one system layer capable of request execution, and we expect it to provide sufficient resources to handle the load.
% Contrarily, for FaaS in a fog environment, a request is forwarded instead of queued, since there are multiple serving layers, and it can not be assumed that a particular one is able to handle all load. 


\section{Auctions} %TODO
\label{sec:auctions}

Not only are FaaS and Fog-Computing dynamic concepts because of the continuous research being done in the respective fields but also due to their core paradigms.
To deal with the constant change these systems are exposed to, it is important to find solutions that are flexible.
To solve the challenges of function placement and load distribution (i.e. where to store a function binary and related data so that it may be executed there) in the heterogeneous, distributed and unreliable environment of Fog Computing, AuctionWhisk employs an auction-inspired approach. 

\subsection{Using auctions for decentralized decision-making}

Auctions offer an approach for making stateless placement decisions on a particular path from the client to the cloud. 
Managing a central source of truth in a distributed system requires much effort and many difficult trade-offs and should hence be avoided if possible. 
Instead, the environment should be able to make decentralized decisions that create a consistent behavior at a high level. 

Auctions can act as such mechanisms since they are characterized through their independence of other systems (e.g., in the real world auctions are mostly uninfluenced by the economy and act as a self-contained market). 
Functions can be efficiently placed with regard to the application's resource needs as well as the cloud providers monetary interests by ranking them by the price a developer is ready to pay. 
Higher bids are executed at the Edge with lower latency while lower bids are executed in closer proximity to the Cloud where resources are more abundant but latency higher. 
This model works particularly well in a fog environment as only little advantage is generated by having shared state among the nodes. 
That is, since the prices should be bound to a location so that individual resource limits are considered. 
For example, the system does not gain an advantage by comparing bids from developers of latency-sensitive applications in (A) a remote resource-constrained region and (B) a metropolitan area with abundant resources, as the scenarios are hardly comparable.

\subsection{Latency effects of auctions}



% Latency can stack up and take not only advantage of fog but provide a disadvantage

Although auctions can save costs for the user while maintaining performance, it can also cause significant performance decrease through auction-induced overhead. 
To determine a winner in an auction, multiple bids must be considered. 
In a quickly evolving environment like FaaS, AuctionWhisk achieves this by aggregating an average over all bids in a timeframe and then admitting those function that offered higher bids than the average. 
This potentially introduces additional latency at two layers. 
Firstly, as the window is aggregated over a certain time period before the winners are determined, functions that arrive early in a window inherit additional latency by having to wait until the window is closed and a decision is made. 
Secondly, if a function should be eliminated during an auction, the call is propagated to the next layer on the path where an auction is held again.
Consequently, the effect can be amplified due to unfortunate timing or an insufficient bid. 
In a case like this, it is likely that there is not only no advantage gained by using a fog infrastructure, but that performance is worse than it would have been, routing the request directly to the cloud. 
That is, since on top of network latency, auction latency is introduced as well. \footnote{Auction latency also includes serialization, deserialization, IO-operations etc.}
This can of course have an impact on the perceived performance-cost ratio of the client.


% Rephrase this better -> maybe this also belongs in my own part?
% For the aforementioned established deployments this means, that assumptions on resources can be made and one of two cases is expected: 
% Either enough resources are available for a request to be executed instantly or requests are queued. From an architecture perspective this is reasonable since there is only one system layer capable of request execution, and we expect it to provide sufficient resources to handle the load.
% Contrarily, for FaaS in a fog environment, a request is forwarded instead of queued, since there are multiple serving layers, and it can not be assumed that a particular one is able to handle all load. 


\section{Use-cases} %TODO
\label{sec:usecases}

The trade-offs AuctionWhisk makes, enable it to work well in some use cases, whilst potentially performing poorly in others. 
The following sections summarize the systems advantages and disadvantages by introducing two examples: On the one hand a scenario AuctionWhisk is optimally suited to handle, and on the other hand a situation that  that does not perform well. To consider only comparable alternatives and draw a relevant conclusion, we will only consider alternatives that are also based on the FaaS paradigm.

\subsection{A}

In the first instance, we will observe an environment where latency is paramount and other aspects, such as cost

\subsection{}

stable latency / performance


\section{Conclusion}
\label{sec:conclusion}

AuctionWhisk adds a new dimension to Fog Computing and the FaaS paradigm by introducing auctions.
This extension has the potential to increase reliability in unreliable Fog systems. Moreover, it offers clients of such a system more control over cost in relation to performance and gives them flexibility through bidding strategies. 
However, auctions also introduce more complexity into the system and counteract the common FaaS promise of an easily useable platform. 
In addition, by being strictly stateless, auctions do not offer sophisticated function scheduling on a global level which worsen resource utilization and efficiency.

\balance

\bibliographystyle{IEEEtran}
\bibliography{bibliography}

\end{document}
